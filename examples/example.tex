\documentclass{article}
\usepackage{amsmath}
\usepackage{graphicx}

\title{A Sampel Document with Speling Erors}
\author{John Doe}
\date{\today}

\begin{document}

\maketitle

% This is a coment with a typo: calcuate
% Another commnet with wrng spellingg

\section{Introducton}

This is an exampl of a LaTeX document with intentinal spelling errors.
The purpse of this docment is to test the spel checker functionality.

\subsection{Bakground}

In this secion, we will discuss some importnt concepts.
The folowing list contains severl items:

\begin{itemize}
    \item First itm with a typo: algoritm
    \item Second itm about mathmatical formulas
    \item Third itm discussing resarch methods
\end{itemize}

\section{Methology}

The methology section describs the approach used in this reserch.
We will demonstate several techniqes for procesing data.

\subsection{Data Colection}

Data was colected from varous sources including:
\begin{enumerate}
    \item Primry sources
    \item Secondry sources
    \item Teriary references
\end{enumerate}

\section{Resuts}

The resuts show that our aproach is efective.
We can see this in the folowing equation:

\begin{equation}
    E = mc^2 \label{eq:enrgy}
\end{equation}

As shown in Equation~\ref{eq:enrgy}, the relatioship is clear.

\subsection{Analyss}

The analyss reveals severl interesting paternzs.
These paterns suggest that our hypothsis was corect.

% Comment: More detals needed here

\section{Concluson}

In concluson, this docment demonstrates variuos spelling errors
that should be detectd by the spell checker. The implimentation
should be able to identfy errors in text, coments, and labels.

\bibliographystyle{plain}
\bibliography{refereces}

\end{document}
